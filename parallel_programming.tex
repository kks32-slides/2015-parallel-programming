\documentclass[10pt,times]{beamer}
\usepackage{amsfonts}
\usepackage{amsmath}
\usepackage{amssymb}
\usepackage{mathptmx}

\usepackage{color}
\usepackage{minted}
\usepackage{hyperref}
\usepackage{multicol}
\usepackage{multirow}
\usepackage{tabularx}
\usepackage{booktabs}
\usepackage{menukeys}
\usepackage{subcaption}

% ******************************** Meta-data ***********************************
\mode<presentation>
{
  \usetheme{Madrid}
  \setbeamercovered{transparent}
}


\usepackage{caption}
\captionsetup{font=scriptsize, labelfont=scriptsize, justification=centering}

\title{An introduction to heterogenous computing}

\author {Krishna Kumar \inst{*}\thanks{github.com/kks32} }

\institute[ University of Cambridge ] % (optional, but mostly needed)
{
%  \includegraphics[width=0.9\textwidth]{figs/goto.png}
}

%\pgfdeclareimage[height=0.2cm]{uni}{figs/Engineering.png}
% \logo{\pgfuseimage{uni}}

\date[\today]{\today}
% Delete this, if you do not want the table of contents to pop up at
% the beginning of each subsection:
\AtBeginSection[]
{
  \begin{frame}<beamer>{Outline}
    \tableofcontents[currentsection,currentsubsection]
  \end{frame}
}


% If you wish to uncover everything in a step-wise fashion, uncomment
% the following command: 

%\beamerdefaultoverlayspecification{<+->}

\subtitle{Accelerating numerical codes}
%***************************** Title page **************************************
\begin{document}
\begin{frame}
  \titlepage
\end{frame}

%*******************************************************************************
%******************************* Frame *****************************************
%*******************************************************************************
\section{Overview: Architecture}

\begin{frame}{von Neumann Architecture}
\begin{columns}
\column{.6\linewidth}
\begin{itemize}
\item Hungarian mathematician John von Neumann circa~1940 - the general requirements 
for an electronic computer.
\item ``Stored-program computer" - both program instructions and data are kept in 
electronic memory.
\begin{itemize}
\item \textit{Read/write}, random access memory is used to store both program 
instructions and data.

\item \textit{Control unit} fetches instructions/data from memory, decodes the 
instructions and then sequentially coordinates operations to accomplish the 
programmed task.

\item \textit{Aritmetic Unit} performs basic arithmetic operations

\item \textit{Input/Output} is the interface to the human operator 
\end{itemize}

\end{itemize} 

\column{.495\linewidth}
\begin{figure}
\includegraphics[width=0.7\linewidth]{figs/vonNeumann}
\caption*{Basic architecture}
\end{figure}
\end{columns}
\end{frame}


%*******************************************************************************
%******************************* Frame *****************************************
%*******************************************************************************

\begin{frame}{Serial computing}

\begin{itemize}
\item Traditionally, software has been written for serial computation:
\begin{itemize}
\item A problem is broken into a discrete series of instructions
\item Instructions are executed sequentially one after another
\item Executed on a single processor
\item Only one instruction may execute at any moment in time 
\end{itemize}

\end{itemize} 

\begin{figure}
\includegraphics[width=0.7\linewidth]{figs/serial}
\caption*{}
\end{figure}
\end{frame}


%*******************************************************************************
%******************************* Frame *****************************************
%*******************************************************************************

\begin{frame}{Memory model}
\begin{columns}
\column{0.49\textwidth}
\begin{figure}
\includegraphics[width=.7\linewidth]{figs/ideal_memory_model}
\caption*{Ideal memory model: \\ We write for this architecture}
\end{figure}

\column{0.49\textwidth}
\begin{figure}
\includegraphics[width=0.9\linewidth]{figs/real_memory_model}
\caption*{Real memory model: How it actually looks}
\end{figure}
\end{columns}

\centering
\textit{The underlying assumption is cache coherency!}
\flushleft
\small
In a shared memory multiprocessor with a separate cache memory for each 
processor, it is possible to have many copies of any one instruction operand: one 
copy in the main memory and one in each cache memory. When one copy of an operand is 
changed, the other copies of the operand must be changed also. Cache coherency 
ensures that changes in the values of shared operands are propagated throughout the 
system in a timely fashion.
\end{frame}


%*******************************************************************************
%******************************* Frame *****************************************
%*******************************************************************************
\begin{frame}{What are caches}
\begin{columns}
\column{0.49\textwidth}
\begin{figure}
\includegraphics[width=0.9\linewidth]{figs/CPU_DRAM}
\caption*{CPU vs DRAM}
\end{figure}

\column{0.49\textwidth}
\begin{figure}
\includegraphics[width=.7\linewidth]{figs/cache}
\caption*{Cache}
\end{figure}
\end{columns}
\begin{itemize}
\item CPU caches are small pools of memory that store information the CPU is most 
likely to need next.

\item A cache miss means the CPU has to go scampering off to find the 
data elsewhere. This is where the L2 cache comes into play — while it's slower, it's 
also much larger. 

\item If data can't be found in the L2 cache, the CPU continues down the chain to L3 
(typically still on-die), then L4 (if it exists) and main memory (DRAM).

\end{itemize}
\end{frame}

\end{document}